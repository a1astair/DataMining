\documentclass[conference]{IEEEtran}
% Some Computer Society conferences also require the compsoc mode option,
% but others use the standard conference format.
%
% If IEEEtran.cls has not been installed into the LaTeX system files,
% manually specify the path to it like:
% \documentclass[conference]{../sty/IEEEtran}

% *** CITATION PACKAGES ***
%
\usepackage{cite}

% correct bad hyphenation here
\hyphenation{op-tical net-works semi-conduc-tor}
 
\begin{document}

%
% paper title

\title{Team Prediction in the English Premier League\\ {\large CSC 587}}
% author names and affiliations


\author{
\IEEEauthorblockN{Alastair Beaumont}
\IEEEauthorblockA{University of Victoria\\
Email: a.beaumont11@gmail.com}\\   %<------ Line breaks in the current column
\IEEEauthorblockN{Kolby Chapman}
\IEEEauthorblockA{University of Victoria\\
Email: kol.j63@gmail.com}
\and
\IEEEauthorblockN{Nishant Khanna}
\IEEEauthorblockA{University of Victoria\\
Email: nishantnkhanna@gmail.com}\\
\IEEEauthorblockN{Graeme Nathan}
\IEEEauthorblockA{University of Victoria\\
Email: graemenathan@gmail.com}
\and
\IEEEauthorblockN{Cole Peterson}
\IEEEauthorblockA{University of Victoria\\
Email: colpeterson@gmail.com}                 %<-----------

}

% make the title area
\maketitle
% As a general rule, do not put math, special symbols or citations
% in the abstract
% For peerreview papers, this IEEEtran command inserts a page break and
% creates the second title. It will be ignored for other modes.
\IEEEpeerreviewmaketitle



\section{Introduction}
This project will mine English Premier League data to predict the winners of soccer games. One factor we are particularly excited to consider is the threat of relegation, as it is a rather unique and important element of the league. We intend to join information from multiple data sources and test many different algorithms on the project, as many freely available dataset are lacking, and many different algorithms have shown success on similar problems.

\section{Related work}

Predicting sports performance is a well studied field. Recently, fantasy sport websites like FanDuel and Draft Kings have opened up new gambling opportunities where machine learning techniques can be applied to predict positive returns\cite{Sugar:aa}, and there has been increased interest in the area\cite{Bishop:2004aa} as the industry grows and becomes more profitable. Sports have valued data mining techniques as it helps eliminate the high emotional stakes the field carries for many of its experts which tend to bias predictions\cite{Haghighat:2013aa}. Sport results can have tremendous world-wide impact, the results of soccer matches have even been shown to affect stock investor behaviour\cite{Edmans:2007aa}, and so accurate prediction can be incredibly valuable outside of the stadium. Most research has investigated predicting the outcome of matches, but now there has been emphasis on selecting a fantasy lineup. Other areas of research include using biomechanical measures to gauge player health and fitness to negotiate contract deals\cite{Davenport:2010aa}, and gauge injury risk \cite{Carling:2012aa,Owen:2015aa}, or examine the impact of mid-season changes to the coaching staff\cite{coach}, all of which can help make better predictions of game outcome, or select better players for a lineup.

Sports data exhibits interesting paradoxes where classic gambling fallacies such as the ``hot-hand'' fallacy actually turn out be real, due to the psychological element of sport. Although the authors of \cite{Gilovich:1985aa} found that in general making one basketball shot did not increase your chances of making another, some baseball players were found to be more ``streaky'' than could be statistically accounted for in \cite{Albert:2008aa}. Making more accurate predictions may rely upon teasing apart some of these more complicated and unintuitive elements of sport data, and verifying conventional wisdom like ``home-field advantage'' is legitimate\cite{textbook}. Identifying the right features to use in training a machine learning model for sport can be tricky. Some approaches have relied upon expert opinion to select features\cite{Zdravevski:2010aa}, while others have added single features iteratively, only keeping them if they increased prediction accuracy\cite{Buursma:2011aa}. Within the game of basketball, a team's performance in the area under the net (short shots and rebounds) was found to have the higher impact on final result than performance anywhere else on the court\cite{Ivankovic:2010aa}. As statistics of soccer tend to be relatively sparse compared to other sports, much effort to apply data mining techniques to the game of soccer have relied upon automated processing of video to detect events in matches\cite{Sykora2015,goal_detection}, player speed \cite{Redwood-Brown:2012aa}, or more complex data like player position on the pitch to discover offensive patterns \cite{offence_patterns}. 

 Many classification techniques have been experimented with including Support Vector Machines \cite{Cao:2012aa}, Neural Networks\cite{McCabe:2008aa}, Bayesian Methods\cite{Buursma:2011aa}, Logistic Regression\cite{Cao:2012aa,Buursma:2011aa}, Fuzzy Models\cite{Trawinski:2010aa}, Decision Trees\cite{Zdravevski:2010aa}, and Markov chain Monte Carlo Methods \cite{Rue:2000aa}. It can be difficult to compare the success of these models in different studies because most of them use different data. Additionally, some studies aim to predict binary win/loss, whereas others \cite{Rotshtein:2005aa} classify with margin of victory granularity. Many different methods have shown to have statistically significant predictive powers, with some studies reporting prediction accuracy $>80\%$\cite{Ivankovic:2010aa}. There remains no consensus or conclusion over which methods provide the highest accuracy for sports in general, or the game of soccer. One study \cite{Cao:2012aa} found logistic classifiers, naive bayes, support vector machines, and artificial neural networks to all produce accuracy of $66.82\% \pm 1.00\%$, demonstrating that no method is clearly better over another on the same data.


%This \cite{Zdravevski:2010aa} has a pretty good comparison of a lot of different methods with different precisions.

%This \cite{Buursma:2011aa} used fifteen years of soccer in the Netherlands as its data. 



\section{Data description}
The data used for this project will be the team stats of soccer teams. Theses stats will include the win/loss record for the team, which team was the home team, the goals scored by each team in the game, and the team’s place in the standings. Using these stats we can potentially predict the outcome of the match.
The data source for this project is a website called www.football-data.co.uk. This website has a csv file containing the scores of every game in a season. The website contains csv files for many different European leagues. There is a csv file for every season dating back to 1993. There are also other websites that have APIs that can be used if more information is needed

\section{Proposed project}
As outlined in the introduction, our project will be predicting the outcomes of two teams meeting in the English soccer Premier League.

We have two goals. The first goal is to build a classifier that is  able to predict the outcome of a game between to member teams of the Premier League with acceptable accuracy (this level of accuracy will be defined further into the project).

The second goal is the investigation of the distance to relegation datum. Some questions we are interested in answer regarding relegation are:
Does a team’s impending relegation improve their performance?
Does the inclusion of data about closeness to relegation improve the accuracy of a classifier?

We will use multiple data sources. Datasets for soccer are limited. To mitigate this, we will combine data from multiple sources and sets to build our own training and testing sets.

We will also use multiple algorithms. Each algorithm will be run on the same sets, and then have their accuracies compared.  The limiting factors here will be time and creativity; we will implement as many algorithms as we can think of that we have time for.

We will gather the data of previous games for each team over multiple years. We might even branch out to other leagues to acquire as much information as possible to increase our training data. Once we have acquired as much training data as possible we will make predictions using our algorithms for past games.

Finally, we will explore the effects of the distance to relegation datum on both the accuracy of our predictions and the performance of the teams.
\section{Distribution of tasks}
At a high level, our project can be broken into two distinct tasks: data preparation and algorithm preparation.

We will have three team members, Alastair, Kolby, and Nishant, working on data preparation while Graeme and Cole will work on algorithm preparation.
Data preparation entails gathering of data, as well as scrubbing in into a format we can use.

Algorithm preparation entails both the design and implementation of an algorithm to work on the data that the data team will provide.
\clearpage
\section{Estimated timeline}
\begin{center}
\small{\begin{tabular}{|c|c|c|c|c|} 
\hline
    Time & Task & Start-Date & End-Date & Team-Members\\\hline
    {Week 1 ( Jan 25 )} & {Project Proposal} & {25 Jan, 2016} & {27 Jan, 2016} &
    \begin{tabular}{l}
        Alastair\\
        Cole\\
        Graeme\\
        Kolby\\
        Nishant\\
    \end{tabular}\\\hline
    
    {Week 2 ( Feb 1 )} &
    \begin{tabular}{l}
      Going through available\\
      data sets that might be\\
      for our project\\
    \end{tabular}
    & { 1 Feb, 2016 } & { 7 Feb, 2016 } &
    \begin{tabular}{l}
        Alastair\\
        Kolby\\
        Nishant\\
    \end{tabular}\\\hline
    
    {Week 3 ( Feb 8 )} &
    \begin{tabular}{l}
      Extracting useful data\\
      from the data set\\
    \end{tabular}
    & { 8 Feb, 2016 } & { 14 Feb, 2016 } & 
    \begin{tabular}{l}
        Alastair\\
        Kolby\\
        Nishant\\
    \end{tabular}\\\hline
    
    {Week 4 ( Feb 15 )} &
    \begin{tabular}{l}
      Extracting more data and\\
      Arranging the data in the\\
      order in which they will\\
      be required\\
    \end{tabular}
    & { 15 Feb, 2016 } & { 21 Feb, 2016 } & 
    \begin{tabular}{l}
        Alastair\\
        Kolby\\
        Nishant\\
    \end{tabular}\\\hline
    
    {Week 5 ( Feb 22 )} &
    \begin{tabular}{l}
      Writing the Algorithm to fit\\
      the data and try to predict\\
      whatever outcome we want\\
    \end{tabular}
    & { 22 Feb, 2016 } & { 28 Feb, 2016 } & 
    \begin{tabular}{l}
      Cole\\
      Graeme\\
    \end{tabular}\\\hline
    
    {Week 6 (Feb 29 )} &
    \begin{tabular}{l}
      Continue writing the algorithm\\
    \end{tabular}
    & { 29  Feb, 2016 } & { 6 Mar, 2016 } & 
    \begin{tabular}{l}
       Cole\\
       Graeme\\
    \end{tabular}\\\hline
    
    {Week 7 ( Mar 7)} &
    \begin{tabular}{l}
      Testing the algorithm to check\\
      whether it works according to\\
      our requirements\\
    \end{tabular}
    & { 7 Mar, 2016 } & { 14 Mar, 2016 } & 
    \begin{tabular}{l}
      Cole\\
      Graeme\\
      Alastair\\
      Kolby\\
      Nishant\\
    \end{tabular}\\\hline
    
    {Week 8 ( Mar 15 )} &
    \begin{tabular}{l}
      Making some changes to the\\
      algorithm if required\\
    \end{tabular}
    & { 15 Mar, 2016 } & { 21 Mar, 2016 } & 
    \begin{tabular}{l}
      Cole\\
      Graeme\\
    \end{tabular}\\\hline
    
    {Week 8 ( Mar 15 )} &
    \begin{tabular}{l}
      Preparing the In class\\
      presentation\\
    \end{tabular}
    & { 15 Mar, 2016 } & { 21 Mar, 2016 } & 
    \begin{tabular}{l}
      Alastair\\
      Kolby\\
      Nishant\\
    \end{tabular}\\\hline
    
    {Week 9 ( Mar 22 )} &
    \begin{tabular}{l}
      Continue with testing the\\
      algorithm and start preparing\\
      the final project report\\
    \end{tabular}
    & { 22 Mar,2016 } & { 28 Mar, 2016 } & 
    \begin{tabular}{l}
      Cole\\
      Graeme\\
      Alastair\\
      Kolby\\
      Nishant\\
    \end{tabular}\\\hline
    
    {Week 10 ( Mar 29 )} &
    \begin{tabular}{l}
      Finalizing the changes to the\\
      algorithm and completing the\\
      project report\\
    \end{tabular}
    & { 29 Mar, 2016 } & { 1 Apr, 2016 } & 
    \begin{tabular}{l}
      Cole\\
      Graeme\\
      Alastair\\
      Kolby\\
      Nishant\\
    \end{tabular}\\\hline
\end{tabular}}
\end{center}


\nocite{*}

% can use a bibliography generated by BibTeX as a .bbl file
% BibTeX documentation can be easily obtained at:
% http://mirror.ctan.org/biblio/bibtex/contrib/doc/
% The IEEEtran BibTeX style support page is at:
% http://www.michaelshell.org/tex/ieeetran/bibtex/
\bibliographystyle{IEEEtran}
% argument is your BibTeX string definitions and bibliography database(s)
\clearpage
\bibliography{b.bib}
%
% <OR> manually copy in the resultant .bbl file
% set second argument of \begin to the number of references
% (used to reserve space for the reference number labels box)
%\begin{thebibliography}{}
%
%\bibitem{}
%
%
%\end{thebibliography}




% that's all folks
\end{document}


