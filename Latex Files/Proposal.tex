\documentclass[conference]{IEEEtran}
% Some Computer Society conferences also require the compsoc mode option,
% but others use the standard conference format.
%
% If IEEEtran.cls has not been installed into the LaTeX system files,
% manually specify the path to it like:
% \documentclass[conference]{../sty/IEEEtran}

% *** CITATION PACKAGES ***
%
\usepackage{cite}


% correct bad hyphenation here
\hyphenation{op-tical net-works semi-conduc-tor}

\begin{document}

%
% paper title

\title{Team Prediction in the English Premier League\\ {\large SENG 474}}

\author{%
  Alastair Beaumont\\University of Victoria\\a.beaumont11@gmail.com
  \and Kolby Chapman\\University of Victoria\\kol.j63@gmail.com
  \and Graeme Nathan\\University of Victoria\\graemenathan@gmail.com
  \and Cole Peterson\\University of Victoria\\colpeterson@gmail.com
}

% make the title area
\maketitle

% As a general rule, do not put math, special symbols or citations
% in the abstract
% For peerreview papers, this IEEEtran command inserts a page break and
% creates the second title. It will be ignored for other modes.
\IEEEpeerreviewmaketitle

\section{Introduction}
This project will mine English Premier League (EPL) data to predict the winners of soccer games. One factor we are particularly excited to consider is the threat of relegation, as it is a rather unique and important element of the league. We intend to join information from multiple data sources and test many different algorithms on the project, as many freely available data set are lacking, and many different algorithms have shown success on similar problems.

\section{Related work}
Predicting sports performance is a well studied field. Recently, fantasy sport websites like FanDuel and Draft Kings have opened up new gambling opportunities where machine learning techniques can be applied to predict positive returns\cite{Sugar:aa}, and there has been increased interest in the area\cite{Bishop:2004aa} as the industry grows and becomes more profitable. Sports have valued data mining techniques as it helps eliminate the high emotional stakes the field carries for many of its experts which tend to bias predictions\cite{Haghighat:2013aa}. Sport results can have tremendous world-wide impact, the results of soccer matches have even been shown to affect stock investor behaviour\cite{Edmans:2007aa}, and so accurate prediction can be incredibly valuable outside of the stadium. Most research has investigated predicting the outcome of matches, but now there has been emphasis on selecting a fantasy lineup. Other areas of research include using biomechanical measures to gauge player health and fitness to negotiate contract deals\cite{Davenport:2010aa}, and gauge injury risk \cite{Carling:2012aa,Owen:2015aa}, or examine the impact of mid-season changes to the coaching staff\cite{coach}, all of which can help make better predictions of game outcome, or select better players for a lineup.

Sports data exhibits interesting paradoxes where classic gambling fallacies such as the ``hot-hand'' fallacy actually turn out be real, due to the psychological element of sport. Although the authors of \cite{Gilovich:1985aa} found that in general making one basketball shot did not increase your chances of making another, some baseball players were found to be more ``streaky'' than could be statistically accounted for in \cite{Albert:2008aa}. Making more accurate predictions may rely upon teasing apart some of these more complicated and unintuitive elements of sport data, and verifying conventional wisdom like ``home-field advantage'' is legitimate\cite{textbook}. Identifying the right features to use in training a machine learning model for sport can be tricky. Some approaches have relied upon expert opinion to select features\cite{Zdravevski:2010aa}, while others have added single features iteratively, only keeping them if they increased prediction accuracy\cite{Buursma:2011aa}. Within the game of basketball, a team's performance in the area under the net (short shots and rebounds) was found to have the higher impact on final result than performance anywhere else on the court\cite{Ivankovic:2010aa}. As statistics of soccer tend to be relatively sparse compared to other sports, much effort to apply data mining techniques to the game of soccer have relied upon automated processing of video to detect events in matches\cite{Sykora2015,goal_detection}, player speed \cite{Redwood-Brown:2012aa}, or more complex data like player position on the pitch to discover offensive patterns \cite{offence_patterns}. 

 Many classification techniques have been experimented with including Support Vector Machines \cite{Cao:2012aa}, Neural Networks\cite{McCabe:2008aa}, Bayesian Methods\cite{Buursma:2011aa}, Logistic Regression\cite{Cao:2012aa,Buursma:2011aa}, Fuzzy Models\cite{Trawinski:2010aa}, Decision Trees\cite{Zdravevski:2010aa}, and Markov chain Monte Carlo Methods \cite{Rue:2000aa}. It can be difficult to compare the success of these models in different studies because most of them use different data. Additionally, some studies aim to predict binary win/loss, whereas others \cite{Rotshtein:2005aa} classify with margin of victory granularity. Many different methods have shown to have statistically significant predictive powers, with some studies reporting prediction accuracy $>80\%$\cite{Ivankovic:2010aa}. There remains no consensus or conclusion over which methods provide the highest accuracy for sports in general, or the game of soccer. One study \cite{Cao:2012aa} found logistic classifiers, naive Bayes, support vector machines, and artificial neural networks to all produce accuracy of $66.82\% \pm 1.00\%$, demonstrating that no method is clearly better over another on the same data.

\section{Data and Project Description}
The data used for this project will be the team stats of soccer teams. Theses stats will include the win/loss record for the team, which team was the home team, the goals scored by each team in the game, and the team's place in the standings. Using these stats we can potentially predict the outcome of the match.
The data source for this project is a website called 'www.football-data.co.uk'\cite{football_data}. This website has a csv file containing the scores of every game in a season. The website contains csv files for many different European leagues. There is a csv file for every season dating back to 1993. There are also other websites that have APIs that can be used if more information is needed.

As outlined in the introduction, our project will be predicting the outcomes of two teams meeting in the EPL.

We have two goals. The first goal is to build a classifier that is  able to predict the outcome of a game between to member teams of the Premier League with acceptable accuracy (this level of accuracy will be defined further into the project).

The second goal is the investigation of the distance to relegation datum. Some questions we are interested in answer regarding relegation are:
Does a team's impending relegation improve their performance?
Does the inclusion of data about closeness to relegation improve the accuracy of a classifier?

We will use multiple data sources. Data sets for soccer are limited. To mitigate this, we will combine data from multiple sources and sets to build our own training and testing sets.

We will also use multiple algorithms. Each algorithm will be run on the same sets, and then have their accuracy compared.  The limiting factors here will be time and creativity; we will implement as many algorithms as we can think of that we have time for.

We will gather the data of previous games for each team over multiple years. We might even branch out to other leagues to acquire as much information as possible to increase our training data. Once we have acquired as much training data as possible we will make predictions using our algorithms for past games.

Finally, we will explore the effects of the distance to relegation datum on both the accuracy of our predictions and the performance of the teams.
\section{Our Approach}
Findng free to use data sources was difficult. Many sports betting sites do not give out their stats for teams and many sports website either do not have APIs or do not have free to use APIs. We were finally able to find a free to use website that contained all the information we were looking for. This website was descibed in the previous section.

The next thing we had to do was to gather the data so that we can use it. To do this, we created a python program that gathers the data from the csv files. This program allowed us to use the data to make predictions. A Season object was used to gather data from specific seasons. This season object could then the queried to get any stats that are needed. A Predictor object was also created to make the predictions. This Predictor object queried the Season object to get the stats needed to make the prediction for the outcome of the game.

The first thing that we wanted to do after gathering the data was to get a baseline for predictions by using some naive predictors. The first naive predictor will always predict that the home team wins and another will always predict that the away team wins. Another predictor will always pick the team that is higher in the standings to win and another will predict the winner based on the win/loss record of the team's previous games. If a team has won a majority of its last games, we will predict it to win. If a team has lost a majority of its last games we will predict it to lose. If we predict both teams to win or both teams to lose, then we will predict a draw. The last naive predictor that we used always picked the winner based on who the betting odds predicted to win. Since the betting odds are made by professionals, we wanted to compare all our predictors against it to see how well we could do compared to the professionals.

For a machine learning algorithm we used a decision tree. We wanted to compare how well the machine learning algorithm did compared to the naive algorithms. We used Leave One Out Cross Validation where one season was left out as testing data.

\section{Results}
A programming framework has been developed which parses the data so that it is easy to work with. The stats from a given season are loaded into a $\mathtt{Season}$ object, which can be queried for stats for any given team or the league at any date in the season. For example, you can request the number of points that a team has on a given date, or the standings of the league, and the object will ensure that it does not use any future data. The $\mathtt{Season}$ object can be expanded to include more features, and serves as the interface between the raw data and our developed code. Additionally, a $\mathtt{Predictor}$ object has been created which uses features from the $\mathtt{Season}$ object to predict the outcome of matches. At this point, relatively naive predictors have been programmed to determine a baseline. One predictor always chooses the home team, one always chooses the away team. One always predicts the team higher in the standings, and one predicts based only on whether teams had won or lost their last game. The baseline results are shown below over five seasons of EPL soccer.

\begin{center}
  \begin{tabular}{@{} cc @{}}
    \hline
    Predictor & Accuracy \\ 
    \hline
High Points & 0.456599190283 \\ 
    Last Game & 0.381983805668\\ 
    Home Team & 0.43979757085 \\ 
    Away Team & 0.308259109312\\ 
    \hline
  \end{tabular}
\end{center}

It is note worthy how well a home-team predictor performs, rivaling a high-points predictor, and confirming that home field advantage in EPL soccer.

A simple decision tree machine learning approach was also tested, using LOOCV where one season was left out as testing data. Limiting the depth of the tree improved results and avoided over fitting, but the accuracy was not much better than the high-points model.  This indicates that more advanced features need to be used, most obviously score.

\begin{center}
  \begin{tabular}{@{} cc @{}}
    \hline
    Decision Tree & Accuracy \\ 
    \hline
Depth = 2 &0.479028340081\\ 
    Depth = 3 & 0.48012145749\\ 
   Depth = 4 & 0.462874493927 \\ 
    Unlimited Depth & 0.39044534413\\ 
    \hline
  \end{tabular}
\end{center}

\section{Conclusion and Future Work}
Future work will look to create a new points measure which is more accurate at predicting the a team's abilities, which reflects the following intuition: losing to a good team is not as bad as losing to a bad team, and losing by a small margin is not as bad as losing by a large margin, as well as coming up for a model for the influence of the threat of relegation. 
To take into account relegation, we will create a relegation score for each team. An example relegation score would use the following formula which is a ratio between (0,1): 

\begin{center}
	1 - ($\mathtt{points}$/$\mathtt{team\_max\_points}$)
\end{center}

Where $\mathtt{points}$ is the current number of points that the team has so far in the season and $\mathtt{team\_max\_points}$ is the maximum number of points a team can get based on the number of games they have left in the season. The higher the relegation score, the closer the team is to relegation. This relegation score formula may still need to be improved.
We will also want to create an algorithm that determines the outcome of the game by looking at the team's last X number of games. We will want to find an X that will provide the most accurate results. For example, if team A has won 3 of its last 5 games and team B has won 1 of its last 5 games then we will predict that team A will win the game. If it is predicted that both teams will win or both teams will lose, then we will predict a draw for that game. We will run this algorithm for different number of X games to find the X that gives the most accurate results. We will compare these results with the results of the naive algorithms to see if there are any improvements. We will run this algorithm with the relegation score and without the relegation score to see if there are any differences in the outcomes. This will determine if a team being in relegation will affect the outcome of the match. 

\section{Task Breakdown}
Kolby and Alastair would work together to do the write ups for the project reports and help out with some of the code. This sub-team also collected all the data for this project using the website and methods explained in Data Description. This sub-team focused on making sure the group was meeting the proper deadlines and helping out with the coding and machine learning aspects when the other sub-team needed help. Kolby and Alastair also coded the betting odds into the graph so we could have a baseline for our algorithms to be compared to.

Graeme and Cole worked on creating python scripts and training the machine learning algorithms on the data. This sub-team trained our decision tree and also the other predictors and created a graph for the results of these predictors based on our data. These two sub-teams worked very well together as we could easily delegate tasks to each other to meet our deadlines. Since most of our scripts were not too challenging, either Graeme or Cole would be assigned to complete these tasks and then we would all check to make sure it generated the results we needed.

Every member of the team would be involved in the decision for deciding our next step and the work was evenly distributed between each member, if Cole and Graeme was working on more scripts, Alastair and Kolby would be working on adding more functionality to the scripts or documenting the changes in our reports to keep track of what is happening. This project was a success due to the high work ethic of every team member and the separation of tasks into sub tasks that each sub-team could tackle with ease. 

\nocite{*}

% can use a bibliography generated by BibTeX as a .bbl file
% BibTeX documentation can be easily obtained at:
% http://mirror.ctan.org/biblio/bibtex/contrib/doc/
% The IEEEtran BibTeX style support page is at:
% http://www.michaelshell.org/tex/ieeetran/bibtex/
\bibliographystyle{IEEEtran}
% argument is your BibTeX string definitions and bibliography database(s)
\bibliography{b}

% that's all folks
\end{document}


